There are many factors that affect the dispersal of gene within a species across a landscape.  Each
factor may have a large effect on particular statistics such as the average or the maximum
pollination distance.  Each of these statistics as well as the actual factors are difficult if not
impossible to accurately measure in the field.  The simulations studied here provide insight to the
effects of plant density as well as a reasonable model for pollinator movement across the landscape.

The majority of models studying pollination have assumed a purely random diffusion process. There is
clear evidence that there are differences in statistics seen between plants pollinated through wind
dispersal from those pollinated through animal dispersal, see ***.  It is demonstrated here through
an agent based correlated random walk that if an animal is not moving in a purely random fashion,
then important pollination statistics can be dramatically affected. 

As can be seen in the results section the magnitude of turning angle had varying degrees of effects
over different plant densities and therefore pollination patterns predicted by a model assuming a
purely random walk could be vastly different from a model assuming a correlated random walk. For
high plant densities, the effects of correlated random walk was less pronounced than that of low
plant densities, except for the \emph{average weighted diversity of fathers}. In the case of
\emph{average weighted diversity of fathers} the affect of turning angle magnitudes were more
pronounced for high densities. Therefore, although diffusion models for densely populated plant
species may not vary greatly from models that assume a correlated random walk for \emph{average
pollination distance} or \emph{average maximum pollination distance} they will vary significantly
for \emph{average weighted diversity of fathers}. This has the affect of under estimating the
diversity of pollination for high plant densities and animal dispersal as compared to similar plant
densities and wind dispersal.

The variation between correlated random walk and that of a purely random walk is significant at low
plant densities for the statistics such as \emph{average maximum distance}, \emph{average
pollination distance}, and \emph{average maximum pollination distance} and so for the case of low
plant densities the assumption of a purely random walk may lend to bias in the analysis of
pollination. Most studies to date have been conducted on small herbaceous plant species whose
densities tend to be high. Even though most of the animals statistics presented were not greatly
affected by turning angle for high plant densities the average weighted diversity of fathers had was
still greatly affected by the turning angle at these densities, and therefore an assumption of a
purely random walk would be an inappropriate assumption and at any of the densities examined in this
study. Therefore a correlated random walk may be a better approximation to animal movement.

