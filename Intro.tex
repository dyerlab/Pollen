Pollination is a critical component of every ecosystem, is essential to creating and maintaining
diversity and reproduction, and is required for world crop production \cite{KleinEtAl2007}.  
There are two main vectors for pollen movement; passive pollination where individual grains are 
dispersed with through the aid of wind and/or water and activt pollination where an animal (most
commonly an insect).  Given the size of individual pollen grains, direct monitoring of how pollen 
is dispersed across the landscape is impractical.  Several indirect approaches have been developed 
including the use of pollen traps and the application of genetic paternity approaches applied to 
successfully pollinated seeds \cite{BitzerPatterson1967,StreiffEtAl1999}.  While these 
approaches are able to quantify the end result of the dispersal process, they provide no information 
the specifics of the precise transport mechanism, which is critical because different features of 
the landscape have variable permeability to pollen movement \cite{DyerSork2001,DyerEtAl2012}. In 
this study, a model is created to simulate animal-mediated dispersion of pollen between plants. 
These models are used to examine the conseqeunces of movement assumptions and their interactions 
with variation in plant density.

Pollen dispersal studies, for both abiotic and biotic pollen dispersal, have assumed that the 
dispersal process is one of random diffusion.  While this may be a good assumption for passively 
pollinated species, there is little evidence that animals randomly diffuse across the landscape
during pollination \cite{LevinKerster}.  In fact, there are several examples of animal pollinators
exhibiting via \emph{trap line} behavior \cite[e.g., repeated sequential visits to individual 
plants]{OhashiThomson}. Even for pollinator species that do not trapline, their movement patterns
do not resemble pure diffusion \cite{Cresswell03}. One approach being applied to describe animal
movement is through the use of correlated random walk (CRW).  In these models, directionality is 
not random but directionality and distance are based on the distribution derived from the movement 
at the previous time step.  Models based upon CRW can have been applied to a wide range of animal
movement processes across varying ecological contexts \cite{Bartumeus07,Byers01}, using 
deterministic diffusion \cite{Klages}, and fractional Brownian motion \cite{Enriquez} approaches.

In this paper, an agent-based model (ABM) describing pollen movement via animals as a correlated 
random walk (CRW) is introduced. This model consist of agents that interact with each other using 
pre-defined rule sets as they explore their environment. Agent based models allow for simulations 
that consist of a large number of interacting parts that would not be easily constructed otherwise 
\cite{Fioretti05}. Agents can represent things such as people, animals, organizations, etc. that 
interact with each other and their environment. The environment in an ABM can represent things 
such as a spatial domain, or a network in which the agents are connected to each other 
\cite{Gilbert}. Using ABMs, the efficacy of CRW apporaches was examined in a system approximating
plant-pollinator interactions using computer simulations. Two statistics describing pollinator movement
(\emph{average path distance} and \emph{average maximum distance}) were collected and parameters relevant
to the plant reproduction (\emph{average pollination distance}, \emph{average maximum
pollination distance}, and \emph{average weighted diversity of fathers}) are collected and analyzed to
describe how assumptions about agant movement infleunce plant pollination dynamics.

It is shown that bias can be introduced by describing animal movement as a purely random walk. That
is, there is a significant difference between the model outcomes for a purely random walk as
compared to a CRW. Thus, modeling animal mediated pollen dispersal by way of a purely random
diffusion process is likely to result in errors in the approximation of the extent of pollen
dispersal.
