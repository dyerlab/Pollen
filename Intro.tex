Pollination is one of the most critical and complicated biological processes on the planet.  It is
an important part of every ecosystem, and is essential to creating diversity and reproduction within
plant species.  There are numerous different mechanisms by which pollination occurs.  Some of which
are abiotic through movement of fluids such as the wind or water, while others are biotic using
animals, often insects, to move pollen from one plant to another.  Studying this process directly is
generally impossible if not impractical.  There are indirect methods developed to infer how
pollination diffuses genes among plants, see .  

In this study a model is created to simulate the dispersion of pollen between plants where
calculations are done to measure the differences between movement assumptions as well as changes in
plant density.  The model mimics both abiotic and biotic movement in order to compare the two types
of diffusion.  There are numerous factors which make this critical process complex to understand as
well as to study.

Understanding the pollination process may allow for optimization of the number of pollinators used
for crop pollination, thereby reducing cost to farmers. Additionally, a better understanding of the
pollination process can lead to the prevention of cross pollination of genetically modified plant
species and non-genetically modified plant species.

Pollen dispersal studies, for both abiotic and biotic pollen dispersal, have focused primarily on
purely random diffusion processes, while this may be a good assumption for species pollinated by
wind, it is most likely an over simplification for species that are pollinated by animals
\cite{Chan}. A purely random diffusion process in two dimensions accurately predicts pollen
dispersal at a particular time, but only for a purely random walk \cite{Byers01}.

Pollen movement via biotic means may not be a purely random process and therefore would not diffuse
in a purely random fashion. In fact, there are several examples of pollinating animals that exhibit
\emph{trap line} behavior \cite{Chan}. That is, they follow a particular route as they collect
pollen.  Thus the movement of animals as they carry pollen may follow more direct paths and
therefore would not result in a purely random diffusion process \cite{Cresswell03}. Such behaviors
result in dispersal that does not mimic a purely random walk. The movement of animals can be
described as a correlated random walk (CRW), where the correlation is based on the distribution and
magnitude of random turning angles. In this way the previous direction of travel influences the
direction of travel for the next step.

A purely random walk can be used to model a purely random diffusion process such as Brownian motion
\cite{Codling}. While a CRW can be used as a general model of animal movement \cite{Prasad05} and
have been successfully used to explore the movement of animals in varying ecological contexts
\cite{Bartumeus07}. CRW models have been used to model the dispersal of bark beetles, Coleoptera:
Scolytidae \cite{Byers01}, deterministic diffusion \cite{Klages}, and fractional Brownian motion
\cite{Enriquez}.

An agent-based model (ABM) describing pollen movement via animals as a correlated random walk (CRW)
is introduced. ABMs consist of agents that interact with each other and their environment. ABMs
allow for simulations that consist of a large number of interacting parts that would not be easily
constructed otherwise \cite{Fioretti05}. Agents can represent things such as people, animals,
organizations, etc. that interact with each other and their environment. The environment in an ABM
can represent things such as a spatial domain, or a network in which the agents are connected to
each other \cite{Gilbert}. ABMs have been used in modeling racial segregation, supply chain
dynamics, and neural networks \cite{Gilbert}.

Consequences of the CRW and the interaction of animals with plants is examined using computer
simulations. Two animals statistics (\emph{average path distance} and \emph{average maximum
distance}) and three plant statistics (\emph{average pollination distance}, \emph{average maximum
pollination distance}, and \emph{average weighted diversity of fathers}) are presented. Turning
angle and plant density are varied and their effects on animals paths and pollen distribution are
examined.

It is shown that bias can be introduced by describing animal movement as a purely random walk. That
is, there is a significant difference between the model outcomes for a purely random walk as
compared to a CRW. Thus, modeling animal mediated pollen dispersal by way of a purely random
diffusion process is likely to result in errors in the approximation of the extent of pollen
dispersal.
